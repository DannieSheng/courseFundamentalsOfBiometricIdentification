\documentclass[a4paper]{article}
\linespread{1.6}
\usepackage{geometry}
\usepackage{setspace}
\usepackage{amsmath}
\usepackage{amssymb}
\usepackage[pdftex]{graphicx}
\geometry{left=1cm,right=1cm,top=2.5cm,bottom=2.5cm}

\begin{document}
\begin{spacing}{2.0}
\begin{flushleft}\begin{huge}EEE6561  Fundamentals of Biometric Identification   Homework 1\end{huge}\end{flushleft}
\begin{enumerate}
\item 
\begin{enumerate}
\item Based on the issue of cooperation, which refers to the behavior of the user when interacting with the system. For the case of this biometric system installed in the bookstore, the enrolled users are likely to cooperate with the system in order to be recognized accurately. Thus this system is a positive recognition system with cooperative users.
\item When considering overt or covert, this system installed is a overt system-with the hand geometry and four digit PIN being used, the system cannot be covert. Every time the users placing their right hand on a scanner and typing in a four digit PIN, they are aware that there would be a deduction of some amount of money from their bank account. 
\item Regarding to habituated users or non-habituated users, this is a system with non-habituated users. The reason is that the users do not use this system on a regular basis, and every time they use this system when buying something, issues like providing low quality of hand geometry or forgetting password may happen.
\item Considering attended or unattended operation, both enrollment and recognition operation should be taken into consideration. When it comes to the enrollment operation, there should be a guidance for the process of biometric data acquisition, i.e. an attended operation. When it comes to the recognition operation, i.e. every time the bank account is accessed and updated, sometimes there may still need the attendance of guidance, especially for those users not so familiar with the system and first-time users. For those users familiar with the system, the system is unattended.
\item This system is used indoor, where most of the ambient environment conditions can be moderated, thus it is a controlled system.
\item When considering whether an open or closed system, this biometric system is only used in access and update the users' bank account, and will not be used on other application, thus it is a closed system.
\end{enumerate}

\newpage
\item I would like to talk about biometric systems deployed in medical services, to be specific, use of iris for accurate patient identification, provided by a company called RightPatient(and others), in the example the technology is used by a hospital called Martin Health System in Stuart, FL.
\begin{enumerate}
\item When it comes to the classification of this system, several considerations should be made:
\begin{enumerate}
\item Based on the issue of cooperation, which refers to the behavior of the user when interacting with the system. For the case of this biometric system in the hospital, the enrolled users are likely to cooperate with the system in order to be recognized accurately. Thus this system is a positive recognition system with cooperative users.
\item When considering overt or covert, this system deployed in Martin Health System is a overt system.
\item Regarding to habituated users or non-habituated users, this system is one with non-habituated users.
\item Considering attended or unattended operation, both enrollment and recognition operation should be taken into consideration. When it comes to the enrollment operation, there should be a guidance for the process of biometric data acquisition, i.e. an attended operation. When it comes to the recognition operation, it is still an attended operation considering sometimes the patient need some guide.
\item This system is mostly used indoor, where most of the ambient environment conditions can be moderated, thus it is a controlled system.
\item When considering whether an open or closed system, in this example the system is used only for the patient identity management, thus it is a closed system.

\end{enumerate}
\item The iris of persons are used for identification. Which is a physical traits.
\item The steps of the operation of the biometric system are as follows:
\begin{enumerate}
\item Patient presents for biometric authentication. His/Her iris structure is scanned. This scanned biometric is then analyzed digitally and converted to its equivalent biometric template.
\item The biometric template obtained in the step above is checked for a match against all the patient records. If a match is found, then the patient is mapped to his existing record in the database and treatment begins. If no match is found, other steps are necessary.
\item If there being no match for the patient in the record, then the patient is identified for enrollment. All of his details such as name, date of birth, address etc. are captured. Along with these details, his biometric template created in step 1 is also stored in the database. The patient is subsequently admitted for treatment.
\end{enumerate}

\item The target population for this system for now is only ones go to the Emergency and Admissions Departments of the hospital, and potentially for future the target population would be everyone. Because everyone should have a unique medical record.

\item Martin Health uses this system because it faces many of the same patient identification challenges which are common to almost all hospitals, such as duplicate medical records, overlays, patient fraud, patient identity theft, etc. The biometric-based system has the advantages listed below, which would solve the problems discussed above:
\begin{enumerate}
\item It discourages fraud. As stated above, patient fraud is a big problem in the medical system, with the iris-based biometric system, the rate of fraud would be reduced to a very low value.
\item It enhances security. The medical record for a person can only be accessed when the iris is scanned, thus it protects the medical privacy.
\item Since the biometrics cannot be easily transferred, forgotten, lost or copied, it benefits the patient when he/she is forgetful or he/she is in too bad a physical conditions to remember something or even talk or write.
\item It eliminates repudiation claims. The iris is unique for a person, so there will be no repudiation.
\item It imparts convenience, which a very important part in this example: it is easy for staff to use, it is hygienic compared traditional ways, and the process is really fast.
\end{enumerate}
\end{enumerate}
reference: \\
http://www.rightpatient.com\\
(under the "CASE STUDIES" page and there is one for "Martin Health", you may need first download it to view it)
\end{enumerate}
\end{spacing}
\end{document}